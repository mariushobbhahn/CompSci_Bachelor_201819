\chapter{Conclusion}
\label{chap:conclusion}

This paper extended the Scattering Transform from classification to object detection. To achieve this two techniques were used. The first is an extension of the hybrid scattering networks used in \cite{ScalingTheScatteringTransform2017} where a scattering network and a CNN are combined sequentially. The second is a new technique introduced in this work where the results from a scattering network are merged with a CNN architecture at different stages. The evaluation of the new techniques was done on seven different datasets. Two of which, PASCAL VOC and KITTI, show real world image scenarios. The other five, Toy\_data, Deformation\_data, Rotation\_data, Translation\_data and Scale\_data, are created just for this work to test performance on images who contain objects with specific properties. Besides the experiments on these datasets the techniques are also tested on very small datasets with very low number of training epochs. Lastly their forward pass is timed and compared to see whether the static nature of the filter yields significant decrease in training time. \\
The results of all those experiments yield many conclusions. The first conclusions can be seen when looking at Figure \ref{fig:comparison} in the comparison Subsection \ref{subsec:comparison_results}. It shows that both scattering approaches match the performance of the standard approach in five out of seven cases, underperform in one and overperform in the other. This is likely due to the scattering transform handling geometric properties or transformations of objects better than the standard approach but having difficulties with objects in high noise environments such as real world datasets outside of controlled environments. \\
When looking at the results of the small data experiments one can see that the sequential scattering approach outperforms the standard and parallel scattering approach for the toy data but both scattering architectures get beaten by the standard architecture for the PASCAL VOC datasets. This again indicates that the sequential scattering approach has an advantage for small datasets and short training times when using data that contain objects with some geometric properties. \\
When comparing the timing, the sequential scattering is faster than the standard approach by taking only 75\% as long per forward pass. The parallel scattering approach is significantly slower while not adding any additional accuracy.\\
Overall this leads to two clear conclusions. First, the parallel scattering approach as described in this paper is in no way better than any of the other two and therefore needs to be part of future research before being used for applications. Second, the sequential scattering approach is neither strictly better than the standard approach nor should it be applied in all cases. However, in some applications it might provide significantly more accurate and more robust results. When the data does not contain much natural noise and has certain geometric patterns the sequential scattering is likely better than the standard approach. Especially when only few data are available sequential scattering is more optimal. This is further amplified in scenarios in which time is a key bottleneck because the time for a forward pass is lower for the sequential scattering approaches. Scenarios where these properties are given could include images of rare diseases since few images exist or object recognition on text because of the clear shapes of characters. 

\section{Future Outlook}

Given the just presented conclusion there are two main avenues for future work. First, additional experiments which support the strength of the scattering transform should be conducted. These could entail training on very small dataset from real applications instead of just creating toy data such as in this paper or finding other use cases that fulfill the specific advantages of the scattering transform, i.e. low noise environments and clear geometric properties of the shapes. The second avenue is an investigation of other ways to set up the parallel scattering network. The setup chosen in this paper is too time consuming and does not achieve a justifiable accuracy compared to the other presented methods. A first step could be to not use the second order scattering coefficients but only first order coefficients instead. 