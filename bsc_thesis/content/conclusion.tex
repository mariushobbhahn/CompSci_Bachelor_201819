\chapter{Conclusion}
\label{chap:conclusion}

This chapter gives a very small summary and an overall conclusion of the techniques and experiments in this paper. Additionally it features an outlook for future work related to or building up on this paper. \\


\section{Future Outlook}

Given the just presented conclusion there are two main avenues for future work. First, additional experiments which support the strength of the scattering transform should be conducted. These could entail training on very small dataset from real applications instead of just creating toy data such as in this paper or finding other use cases that fulfill the specific advantages of the scattering transform, i.e. low noise environments and clear geometric properties of the shapes. The second avenue is an investigation of other ways to set up the parallel scattering network. The setup chosen in this paper is too time consuming and does not achieve a justifiable accuracy compared to the other presented methods. A first step could be to not use the second order scattering coefficients but only first order coefficients instead. 