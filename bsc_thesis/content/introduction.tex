\chapter{Introduction}
\label{chap:introduction}

Object detection describes the task of detecting instances of semantic objects in visual data, i.e. images and videos in two or three dimensions respectively. Even though the task is very easy for humans in most situations it is very hard for computers. However, in recent years object detection algorithms have gotten significantly better for many different applications like face recognition, object tracking (e.g. the ball in a football match) and especially semantic segmentation of traffic scenes, pedestrian and car tracking. 
%need references for algorithms and use cases

For most state of the art (SOTA) object detection algorithm convolutional neural networks (CNNs) are used. In many implementations the filters used in those CNNs are all trained during the training period. \cite{scatteringTransform2012}
introduced a new technique called the Scattering Transform that uses wavelet operations on the image and performs classification tasks on those. They also show that the technique is essentially equivalent to using CNNs with fixed weights for some or all filters. It has been applied successfully in a variety of tasks. \cite{InvariantScatteringTextureDiscrimination2013} showed that the scattering transform is applicable to texture discrimination. \cite{DeepRotoTranslation2014} have demonstrated that the scattering transform also produces results similar to other SOTA algorithms for unsupervised learning. \cite{3DScatteringTransformNeuro2017} improved the classification of diseases from neuroimages considerably. Lastly, \cite{ScalingTheScatteringTransform2017} shows that substituting the first layer filters of CNN approaches with the scattering transform yields equivalent results compared to these filters being trained.\\
The reason why the scattering transform has proven so successful are the properties it provides. It is invariant to deformation and equivariant to translation. In this work it is also argued that it has desirable properties w.r.t rotation and scaling. These properties are important for image classification but also necessary for object detection. For example, when detecting pedestrians in real traffic situations, the object detection algorithm must be able to identify them independent of their location, size or rotation within the image. \\
This work tries to harvest the useful properties of the scattering transform and combine it with already established state of the art object detection algorithms. This will be done primarily in two ways. First, the techniques are combined sequentially, i.e. the SOTA algorithms are applied only to the outputs of the scattering transform. \cite{ScalingTheScatteringTransform2017}
have already shown that sequential combination is able to produce SOTA results for image recognition. This is the attempt to reproduce these findings for object detection. \\
Second, the techniques are combined in parallel, i.e. the information of the scattering transform are used as additional inputs for the object detection algorithm or merged at later stages. This has not been tested yet and is the primary extension of the just described related work.\\
If the approaches are successful, three specific advantages arise. First, the scattering transform might yield information that was currently not available to the network and therefore increasing its accuracy. Even if that might only be a marginal increase, it is meaningful for application. Every little reduction of the error in object detection, especially for autonomous driving, means a reduction of risk of self driving cars. This is directly translated to lives being saved in the longterm. 
Second, fixed weights imply no additional training time for them. If, for example, one layer can be substituted that would reduce the length of training and save cost and energy while creating access for people who currently do not own multiple GPUs.
Third, fixed weights cannot be overfit and are maximally general. This might produce more robust algorithms and protect against black box attacks or other malicious practices applied to CNNs. This, however, will not be tested within the scope of this work but might be interesting follow-up.