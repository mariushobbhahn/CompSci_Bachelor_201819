\chapter*{Abstract}
\label{chap:abstract}

The Scattering Transform is a filter that can be applied on images with local invariance but global equivariance w.r.t. translation, scale, rotation and deformation of an object. It has already been shown that it is beneficial to use the scattering transform in combination with Deep Learning architectures in the case of classification. Specifically if there are low number of training data available or the data has specific geometric properties the Scattering Transforms is very useful. This paper extends the use of the Scattering Transform from classification to object detection by combining it with classical object detection architectures such as an SSD in two ways. The first is analogous to the classification case, namely combining the architectures sequentially. The data is first transformed by the Scattering Transform and the Deep Learning architecture learns based on the resulting scattering coefficients. The second is a technique that is new to this paper. The Scattering Transform and the Deep Learning architecture are combined in parallel, that is the data is fed through both methods and then continuously merged at later stages of the network. The respective methods are tested and compared to the conventional approach on different datasets. Some of the datasets are created specifically to test the attributes that the theoretical guarantees of the Scattering Transform would predict, i.e. transformations such as scaling, deforming, rotating or translating an object. Lastly, it is tested on small datasets with low amounts of training and the timing of a forward pass is compared to other approaches. The experiments conclude that the Scattering Transform if combined sequentially is able to achieve similar results to the standard approach in most cases, underperforms in noisy environments such as real life images but overperforms when small amounts of data is available and the objects have specific geometric properties. This makes it a useful tool for specific applications.

\chapter*{Zusammenfassung}
\label{chap:zusammenfassung}

Die Scattering Transformation ist ein Filter, der auf Bilder angewendet werden kann. Sie hat die Eigenschaft lokal invariant aber global equivariant im Bezug auf Translation, Skalierung, Rotation und Deformation zu sein. Es wurde in früheren Arbeiten bereits gezeigt, dass diese Eigenschaft vorteilhaft für die Klassifizierung von Objekten auf Bildern ist wenn man die Scattering Transformation mit anderen Deep Learning Architekturen kombiniert. Besonders wenn wenig Datenpunkte oder bestimmte geometrische Eigenschaften der Objekte vorhanden sind, ist die Scattering Transformation erfolgreich. Diese Arbeit erweitert die Scattering Transformation vom Bereich der Klassifizierung auf die Objekterkennung in zwei Arten. Zunächst, analog zur Klassifizierung, werden die Scattering Transformation und die Deep Learning Architektur sequentiell kombiniert. Das heißt, die Daten werden erst von der Scattering Transformation verändert und deren Ergebnis dann in die konventionelle Architekture gefüttert. Die zweite Methode ist neu und wird in dieser Arbeit eingeführt. Dabei werden die beiden Architekturen parallel kombiniert. Die Daten werden sowohl durch die Scattering Transformation als auch durch die herkömmliche Deep Learning Architektur verarbeitet und dann kontinuierlich verschmolzen. Die Methoden werden auf verschiedenen Datensätzen getestet und mit den konventionellen Ansätzen verglichen. Dabei sind manche Datensätze spezifisch für das testen von den theoretisch vorhergesagten Eigenschaften der Scattering Transformation erschaffen worden. Zusätzlich werden auch sehr kleine Datensätze mit wenig Trainingszeit getestet und die Zeit die der jeweilige Ansatz für das verarbeiten eines Bilder benötigt gemessen und verglichen. Die Experimente zeigen, dass die Scattering Transformation, im Fall der sequentiellen Kombination, gleich gute Resultate wie die konventionellen Ansätze aufweisen. Im Fall von Daten mit viel Hintergrundrauschen ist sie etwas schlechter, im Fall von Translation der Objekte wesentlich besser. Wenn nur wenige Daten vorhanden sind und die Objekte spezifische geometrische Eigenschaften haben ist die Scattering Transformation ebenfalls überlegen. Damit ist sie ein sinnvolles Werkzeug für manche Anwendungen. 